\section*{Model following control}
\addcontentsline{toc}{section}{Model following control}

\subsection*{Level 1}
\addcontentsline{toc}{subsection}{Level 1}


Here, we will design a model following controller by inverting the process model in the time domain.

The control structure is depicted in Figure \ref{contStruct}.

\begin{figure}[ht]
\begin{center}
$$\text{TO BE CREATED}$$
\end{center}
 \caption{Control structure for the DC-motor}
 \label{contStruct}
\end{figure}

The model following block is depicted in Figure \ref{modelFollow}.

\begin{figure}[ht]
\begin{center}
$$\text{TO BE CREATED}$$
\end{center}
 \caption{Model following for the DC-motor}
 \label{modelFollow}
\end{figure}

Since the DC-motor model we used is a second order system, the reference position must be two times differentiable. 

The trajectory planner is design using the fastest possible positionning:
\begin{equation}a_{max} = \frac{\pm M_{max}}{J}\end{equation}
\begin{equation}v_{max} = \frac{U_{max} - \frac{R}{k_\varphi} F_c}{\frac{R d}{k_\varphi} + k_\varphi}\end{equation}

With:

$M_{max}$ : maximum torque of the motor

$v_{max}$ : maximum reachable velocity of the DC-motor, computed with the DC motor model equations.





